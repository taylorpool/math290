\documentclass[12pt,oneside]{article}

% This package simply sets the margins to be 1 inch.
\usepackage[margin=1in]{geometry}

% These packages include nice commands from AMS-LaTeX
\usepackage{amssymb,amsmath,amsthm}

% Make the space between lines slightly more
% generous than normal single spacing, but compensate
% so that the spacing between rows of matrices still
% looks normal.  Note that 1.1=1/.9090909...
\renewcommand{\baselinestretch}{1.1}
\renewcommand{\arraystretch}{.91}

% Define an environment for exercises.
\newenvironment{exercise}[1]{\vspace{.1in}\noindent\textbf{Exercise #1 \hspace{.05em}}}{}

% define shortcut commands for commonly used symbols
\newcommand{\R}{\mathbb{R}}
\newcommand{\C}{\mathbb{C}}
\newcommand{\Z}{\mathbb{Z}}
\newcommand{\Q}{\mathbb{Q}}
\newcommand{\N}{\mathbb{N}}
\newcommand{\calP}{\mathcal{P}}

\DeclareMathOperator{\vsspan}{span}

%%%%%%%%%%%%%%%%%%%%%%%%%%%%%%%%%%%%%%%%%%

\begin{document}

% If you use Overleaf, the name of the project will be determined by
% what you enter as the document title.
\title{Math 290 Homework Template}

\begin{flushright}
\textsc{Taylor Pool}  \\
Math 290 Sec 2\\
Date Jan 23, 2017
\end{flushright}

\begin{center}
\textsf{Assignment 1} \\
\textsf{Exercises: 1}
\end{center}

%%%%%%%%%%%%%%%%%%%%%%%%%%%%%%%%%%%%%%%%

\begin{exercise}{1}
Turn in a replica of the following two paragraphs (don't worry about the font size, or getting the line breaks to match up- just produce readable text that says the same thing, with the same appearance.
\end{exercise}

\begin{proof}
The Cartesian product (or simply the product) $A\times B$ of two sets $A$ and $B$ is the set consisting of all ordered pairs whose first coordinate belongs to $A$ and whose second coordinate belongs to $B$. In other words, \[A \times B = \{(a,b):a \in A \text{ and } b \in B\}. \]
For example, if $A=\{x,y\}$ and $B=\{1,2,3\}$, then
\[A \times B = \{(x,1),(x,2),(x,3),(y,1),(y,2),(y,3)\} \text{;}\]
while
\[B \times A = \{(1,x)(1,y),(2,x),(2,y),(3,x),(3,y)\} \text{.}\]

Since, for example, $(x,1) \in A \times B$ and $(x,1) \notin B \times A$, these two sets do not contain the same elements, so $A \times B \neq B \times A$. If $A = \emptyset \text{ or } B = \emptyset$, then $A \times B = \emptyset$.

For the sets $A$ and $B$ just mentioned, $|A|=2$ and $|B|=3$; while $|A \times B|=|B \times A| = 6$. Indeed, for all finite sets $A$ and $B$,
\[|A \times B| = |A| \cdot |B| \text{.}\]
\end{proof}


%%%%%%%%%%%%%%%%%%%%%%%%%%%%%%%%%%%%%%%%

%\begin{exercise}{???}
%The statement of the exercise goes here.
%\end{exercise}

%\begin{proof}[Solution]
%The explanation or solution goes here.
%\end{proof}


%%%%%%%%%%%%%%%%%%%%%%%%%%%%%%%%%%%%%%%%

%\begin{exercise}{???}
%The statement of the exercise goes here.
%\end{exercise}

%\begin{proof}
%The proof goes here.
%\end{proof}


%---------------------------------
% Don't change anything below here
%---------------------------------


\end{document}
