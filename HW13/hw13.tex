\documentclass[12pt,oneside]{article}

% This package simply sets the margins to be 1 inch.
\usepackage[margin=1in]{geometry}

% These packages include nice commands from AMS-LaTeX
\usepackage{amssymb,amsmath,amsthm}

% Make the space between lines slightly more
% generous than normal single spacing, but compensate
% so that the spacing between rows of matrices still
% looks normal.  Note that 1.1=1/.9090909...
\renewcommand{\baselinestretch}{1.1}
\renewcommand{\arraystretch}{.91}

% Define an environment for exercises.
\newenvironment{exercise}[1]{\vspace{.1in}\noindent\textbf{Exercise #1 \hspace{.05em}}}{}

% define shortcut commands for commonly used symbols
\newcommand{\R}{\mathbb{R}}
\newcommand{\C}{\mathbb{C}}
\newcommand{\Z}{\mathbb{Z}}
\newcommand{\Q}{\mathbb{Q}}
\newcommand{\N}{\mathbb{N}}
\newcommand{\calP}{\mathcal{P}}

\DeclareMathOperator{\vsspan}{span}

%%%%%%%%%%%%%%%%%%%%%%%%%%%%%%%%%%%%%%%%%%

\begin{document}

% If you use Overleaf, the name of the project will be determined by
% what you enter as the document title.
\title{Math 290 Homework Template}

\begin{flushright}
\textsc{Taylor Pool}  \\
Math 290 Sec 2\\
Date Feb 14, 2017
\end{flushright}

\begin{center}
\textsf{Assignment 1} \\
\textsf{Exercises: 1}
\end{center}

%%%%%%%%%%%%%%%%%%%%%%%%%%%%%%%%%%%%%%%%

\begin{exercise}{13.1}
Prove that for every $n \in \N$,
\\
\begin{equation*}
\sum_{i=1}^n (2i-1) = n^2.
\end{equation*}
\end{exercise}

\begin{proof}
Let $P(n): \sum_{i=1}^n (2i-1) = n^2$.
\subparagraph*{Base Case:}
Prove $P(1): \sum_{i=1}^1 (2i-1) = 1^2$.\\
$2(1)-1=1$ so
$1=1$, which is true.

\subparagraph*{Inductive Step:}
Assume $P(k): \sum_{i=1}^k (2i-1) = k^2$.\\
Then we need to prove $P(k+1):\sum_{i=1}^{k+1} (2i-1) = (k+1)^2$.\\
We begin on the left-hand side.\\
\begin{align*}
\sum_{i=1}^{k+1} (2i-1) &= \sum_{i=1}^k (2i-1) + 2(k+1)-1 \\
&= k^2+2k+1 \\
&= (k+1)^2
\end{align*}
But $(k+1)^2$ is the right-hand side, so we have proved $P(k+1)$.\\
Thus, by induction, $P(n)$ is true for all $n \in \N$.
\end{proof}

\begin{exercise}{13.2}
Prove that for every $n \in \N$,
\begin{equation*}
\sum_{i=1}^n \frac{1}{(2i-1)(2i+1)} = \frac{n}{2n+1}
\end{equation*}
\end{exercise}

\begin{proof}
\begin{equation*}
\text{Let~} P(n): \sum_{i=1}^n \frac{1}{(2i-1)(2i+1)} = \frac{n}{2n+1}
\end{equation*}

\subparagraph*{Base Case:}
Prove $P(1): \sum_{i=1}^1 \frac{1}{(2i-1)(2i+1)} = \frac{1}{2(1)+1}$.\\
$\frac{1}{(2-1)(2+1)} = \frac{1}{3}$ so $\frac{1}{3} = \frac{1}{3}$, which is true.

\subparagraph*{Inductive Step:}
Assume $P(k): \sum_{i=1}^k \frac{1}{(2i-1)(2i+1)} = \frac{k}{2k+1}$.

Then we need to prove 
\begin{equation*}
P(k+1): \sum_{i=1}^{k+1} \frac{1}{(2i-1)(2i+1)} = \frac{k+1}{2(k+1)+1} \text{~is true.}
\end{equation*}

\begin{align*}
\textit{LHS} = \sum_{i=1}^{k+1} \frac{1}{(2i-1)(2i+1)} &= \sum_{i=1}^k \frac{1}{(2i-1)(2i+1)} + \frac{1}{[2(k+1)-1][2(k+1)+1]} \\
&= \frac{k}{2k+1} + \frac{1}{(2k+1)(2k+3)} \\
&= \frac{k(2k+3)+1}{(2k+1)(2k+3)}\\
&= \frac{2k^2+3k+1}{(2k+1)(2k+3)}\\
&= \frac{(2k+1)(k+1)}{(2k+1)(2k+3)}\\
&= \frac{k+1}{2k+3}\\
&= \frac{k+1}{2(k+1)+1} = \textit{RHS}
\end{align*}

Thus, since $P(k+1)$ is true, we have proved the proposition by mathematical induction.
\end{proof}

\begin{exercise}{13.3}
Prove that for every $n \in \N$,
\begin{equation*}
\sum_{i=1}^n i^2 = \frac{n(n+1)(2n+1)}{6}
\end{equation*}
\end{exercise}

\begin{proof}


\begin{equation*}
\text{Let~} P(n): \sum_{i=1}^n i^2 = \frac{n(n+1)(2n+1)}{6}
\end{equation*}

\subparagraph*{Base Case:}
Prove $P(1): \sum_{i=1}^1 i^2 = \frac{(1)(1+1)(2(1)+1)}{6}$.\\
$1^2 = \frac{2(3)}{6}$, so $P(1)$ is true.

\subparagraph*{Inductive Step:}
Assume $P(K): \sum_{i=1}^k i^2 = \frac{k(k+1)(2k+1)}{6}$.\\
Then we need to prove
\begin{equation*}
P(k+1): \sum_{i=1}^{k+1} i^2 = \frac{(k+1)(k+1+1)(2(k+1)+1)}{6} \text{~is true.}
\end{equation*}


\begin{align*}
\textit{LHS} = \sum_{i=1}^{k+1} i^2 &= \sum_{i=1}^k i^2+(k+1)^2 \\
&= \frac{k(k+1)(2k+1)}{6}+(k+1)^2 \\
&= \frac{k(k+1)(2k+1)+6(k+1)^2}{6} \\
&= \frac{(k+1)[k(2k+1)+6(k+1)]}{6} \\
&= \frac{(k+1)(2k^2+7k+6)}{6} \\
&= \frac{(k+1)(k+2)(2k+3)}{6} \\
&= \frac{(k+1)(k+1+1)(2(k+1)+1)}{6} = \textit{RHS}
\end{align*}

Thus, since $P(k+1)$ is true when $P(k)$ is true, and since $P(1)$ is true, we have proved the proposition by mathematical induction.
\end{proof}

\begin{exercise}{13.4}
Prove that for every $n \in \N$, $n < 3^n$.
\end{exercise}

\begin{proof}
Let $P(n): n < 3^n$.

\subparagraph*{Base Case:}
Prove $P(1): 1 < 3^1$. \\
$1 < 3$, so $P(1)$ is true.

\subparagraph*{Inductive Step:}
Assume $P(k): k < 3^k$ is true. \\
Then we need to prove $P(k+1): k+1 < 3^{k+1}$ is true. \\

\begin{align*}
\textit{RHS} = k+1 < 3^k+1
% Continue here!
\end{align*}
\end{proof}

\begin{exercise}{13.5}
Let $a,x \in \R$, with $x \not=1$. Prove that $\forall n \in \N$,
\begin{equation*}
\sum_{i=0}^n x_i = \frac{1-x^{n+1}}{1-x}
\end{equation*}
\end{exercise}

\begin{proof}
\begin{equation*}
\text{Let~} P(n): \sum_{i=0}^n x_i = \frac{1-x^{n+1}}{1-x}
\end{equation*}

\subparagraph*{Base Case:}
\begin{equation*}
\text{Prove~} P(1): \sum_{i=0}^1 x_i = \frac{1-x^{1+1}}{1-x}
\end{equation*}
\begin{align*}
\sum_{i=0}^1 x_i &= \frac{1-x^{1+1}}{1-x}\\
x^0+x^1 &= \frac{1-x^2}{1-x}\\
1+x &= \frac{1-x^2}{1-x}\\
\frac{(1+x)(1-x)}{1-x} &= \frac{1-x^2}{1-x}\\
\frac{1-x^2}{1-x} &= \frac{1-x^2}{1-x}
\end{align*}
Thus, we have proved the base case.

\subparagraph*{Inductive Step:} 
\begin{equation*}
\text{Assume~} P(k): \sum_{i=0}^k x_i = \frac{1-x^{k+1}}{1-x} \text{~is true.}
\end{equation*}

\begin{equation*}
\text{Prove~} P(k+1): \sum_{i=0}^{k+1} x_i = \frac{1-x^{k+1+1}}{1-x} \text{~is true.}
\end{equation*}

\begin{align*}
\textit{LHS} = \sum_{i=0}^{k+1} x_i &= \sum_{i=0}^k x^i + x^{k+1}\\
&= \frac{1-x^{k+1}}{1-x} + x^{k+1} \\
&= \frac{1-x^{k+1}+(1-x)x^{k+1}}{1-x} \\
&= \frac{1-x^{k+1}+x^{k+1}-x^{k+2}}{1-x} \\
&= \frac{1-x^{k+2}}{1-x} = \textit{RHS}
\end{align*}

Thus, since $P(1)$ is true and $P(k+1)$ is true if $P(k)$ is true, we have proved the proposition.
\end{proof}
%%%%%%%%%%%%%%%%%%%%%%%%%%%%%%%%%%%%%%%%

%\begin{exercise}{???}
%The statement of the exercise goes here.
%\end{exercise}

%\begin{proof}[Solution]
%The explanation or solution goes here.
%\end{proof}


%%%%%%%%%%%%%%%%%%%%%%%%%%%%%%%%%%%%%%%%

%\begin{exercise}{???}
%The statement of the exercise goes here.
%\end{exercise}

%\begin{proof}
%The proof goes here.
%\end{proof}


%---------------------------------
% Don't change anything below here
%---------------------------------


\end{document}
